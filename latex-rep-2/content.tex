\section{Постановка задачи}
Необходимо сгенерировать массивы данных для пяти распределений:
\begin{itemize}
\item нормальное распределение \( N(x, 0, 1) \)
\item распределение Коши \( C(x, 0, 1) \)
\item распределение Лапласа \( L(x, 0, 1/\sqrt{2}) \)
\item распределение Пуассона \( P(k, 10) \)
\item равномерное распределение \( U(x, -\sqrt{3}, \sqrt{3}) \)
\end{itemize}
Для каждого распределения массивы должны состоять из 10, 100, 1000 элементов. Для выборок вычислить статистические характеристики положения данных:
\begin{itemize}
\item выборочное среднее \( \bar{x} \)
\item выборочная медиана \( med x \)
\item полусумма экстремальных значений \( Z_{R} \)
\item полусумма квартилей \( Z_{Q} \)
\item усечённое среднее \( Z_{tr}\)
\end{itemize}

Повторить вычисления 1000 раз для каждой выборки, найти среднее и дисперсию характеристик положения. 

\begin{equation}
D(z) = \bar{z^2} - {\bar{z}}^2
\end{equation}

Представить данные в виде таблиц.



\section{Теория}
\subsection{Выборочное среднее}

\begin{equation}
\bar{X} = \frac{1}{n} \sum\limits_{i=1}^n X_i
\end{equation}


\subsection{Выборочная медиана}

\begin{equation} 
med x = \begin{cases}
   x_{(k+1)}, n = 2k+1\\
   \frac{x_{(k)} + x_{(k+1)}}{2}, n = 2k
 \end{cases}
\end{equation}


\subsection{Полусумма экстремальных значений}

\begin{equation} 
Z_{R} = \frac{x_{(1)} + x_{(n)}}{2}
\end{equation}


\subsection{Полусумма квартилей}

Квартиль \(z_p\) порядка \(p\) определяется формулой:
\begin{equation} 
Z_{p} = \begin{cases}
   x_{([np] + 1)}, \text{при дробном np}\\
   x_{(np)}, \text{при целом np}
 \end{cases}
\end{equation}

\begin{equation} 
Z_{Q} = \frac{z_{1/4} + z_{3/4}}{2}
\end{equation}


\subsection{Усечённое среднее}

\begin{equation} 
Z_{tr} = \frac{1}{n-2r}\sum_{i=r+1}^{n-r} x_{i}
\end{equation}


\section{Реализация}
\begin{itemize}
\item Язык: Python
\item Среда разработки: PyCharm
\item Используемые библиотеки: NumPy, SciPy
\end{itemize}


\section{Результаты}

\subsection{Нормальное распределение}

\begin{table}[H]
\caption{Нормальное распределение}
\begin{center}
 \begin{tabular}{||c || c c c c c||} 
 \hline
 Normal & \(\bar{X}\)  &  med x & Z_{R} & Z_{Q} & Z_{tr} \\ 
 \hline\hline
  E(z), n=10 &  0.01 & 0.0 & 0.0 & 0.3 & 0.0 \\ 
 \hline
  D(z), n=10 & 0.099901 & 0.130724 & 0.188363 & 0.125813 & 0.110472 \\
 \hline
  E(z), n=100 & 0.004 & 0.00 & 0.01 & 0.02 & 0.00 \\
 \hline
  D(z), n=100 & 0.009973 & 0.015257 & 0.086898 & 0.012487 & 0.012003\\
 \hline
   E(z), n=1000 & 0.001 & 0.001 & 0.01 & 0.002 & 0.001 \\
 \hline
  D(z), n=1000 & 0.001005 & 0.001504 & 0.060459 & 0.001231 & 0.001161 \\
 \hline
\end{tabular}
\end{center}
\end{table}

\\
\\
\\
\subsection{Распределение Коши}

\begin{table}[H]
\caption{Распределение Коши}
\begin{center}
 \begin{tabular}{||c || c c c c c||} 
 \hline
 Cauchy & \(\bar{X}\)  &  med x & Z_{R} & Z_{Q} & Z_{tr} \\ 
 \hline\hline
  E(z), n=10 &  1.  & -0.0 & 4. & 1. & 0.0 \\ 
 \hline
  D(z), n=10 & 1822.236796 &  0.323789 &   45560.167455 & 4.599842 & 0.496796 \\
 \hline
  E(z), n=100 & -1. & 0.00 & -36. & 0.04 & 0.01 \\
 \hline
  D(z), n=100 & 289.181485 & 0.025169 & 703406.624051 & 0.052215 & 0.026765\\
 \hline
   E(z), n=1000 & 0. & 0.000 & 155. & 0.004 & 0.000 \\
 \hline
  D(z), n=1000 & 636.054459 &  0.002434 & 149968485.264449 & 0.005171 & 0.002611 \\
 \hline
\end{tabular}
\end{center}
\end{table}

\subsection{Распределение Лапласа}
\begin{table}[H]
\caption{Распределение Лапласа}
\begin{center}
 \begin{tabular}{||c || c c c c c||} 
 \hline
 Laplace & \(\bar{X}\)  &  med x & Z_{R} & Z_{Q} & Z_{tr} \\ 
 \hline\hline
  E(z), n=10 &  0.00 & -0.00 &  0.0 & 0.3 & -0.00 \\ 
 \hline
  D(z), n=10 & 0.098091 & 0.068721 & 0.4208 & 0.114753 & 0.069268 \\
 \hline
  E(z), n=100 & 0.00 &  0.001 & -0.0 & 0.02 & 0.001 \\
 \hline
  D(z), n=100 & 0.01014  &  0.005763 &   0.4357 & 0.010622 & 0.006427\\
 \hline
   E(z), n=1000 & 0.001 &  0.0007 &  0.0 & 0.002 & 0.0004  \\
 \hline
  D(z), n=1000 & 0.001094 &  0.000551 &  0.374109 & 0.001056 &  0.000652 \\
 \hline
\end{tabular}
\end{center}
\end{table}


\subsection{Распределение Пуассона}
\begin{table}[H]
\caption{Распределение Пуассона}
\begin{center}
 \begin{tabular}{||c || c c c c c||} 
 \hline
 Poisson & \(\bar{X}\)  &  med x & Z_{R} & Z_{Q} & Z_{tr} \\ 
 \hline\hline
  E(z), n=10 &   9.9  &  9   &  10 & 10  &  9 \\ 
 \hline
  D(z), n=10 & 0.878343 &1.295096 & 1.858151 & 1.191244 & 1.001723 \\
 \hline
  E(z), n=100 & 9.9  &  9.9   & 10.9  &   9.9  &  9.8 \\
 \hline
  D(z), n=100 & 0.100212 & 0.212336 & 0.93767  &  0.1594  & 0.119614 \\
 \hline
   E(z), n=1000 & 9.996 &  9.995  &  11.6   &  9.997   & 9.85  \\
 \hline
  D(z), n=1000 & 0.009732 & 0.00522  &  0.5851  & 0.001491 & 0.010727 \\
 \hline
\end{tabular}
\end{center}
\end{table}


\subsection{Равномерное распределение}
\begin{table}[H]
\caption{Равномерное распределение}
\begin{center}
 \begin{tabular}{||c || c c c c c||} 
 \hline
 Uniform & \(\bar{X}\)  &  med x & Z_{R} & Z_{Q} & Z_{tr} \\ 
 \hline\hline
  E(z), n=10 &   0.00 & -0.0 &  0.01 & 0.3  & -0.0 \\ 
 \hline
  D(z), n=10 & 0.098251 &  0.226369 &  0.044721 & 0.125385 &  0.160175 \\
 \hline
  E(z), n=100 & 0.00 &  0.00 & -0.0001 & 0.02 &  0.00 \\
 \hline
  D(z), n=100 & 0.010502 &  0.030381 &  0.000544 & 0.015325 &  0.02032 \\
 \hline
   E(z), n=1000 & -0.000 &  0.006 &  -4.3e-05 & 0.001 & -0.000  \\
 \hline
  D(z), n=1000 & 0.001039 &  0.003019 &   5e-06   & 0.001515 &  0.002046 \\
 \hline
\end{tabular}
\end{center}
\end{table}


\section{Обсуждение}
Упорядочим полученные значения для каждого распределения.\\
Для нормального распределения:
\[Z_{tr} < \bar{X} < med X < Z_{Q} < Z_{R}\]
Для распределения Коши:
\[med X < Z_{tr} < Z_{Q} < \bar{X} < Z_{R}\]
Для распределения Лапласа:
\[Z_{tr} < med X < \bar{X} < Z_{Q} < Z_{R}\]
Для распределения Пуассона:
\[Z_{tr} < med X < \bar{X}  < Z_{Q} < Z_{R}\]
Для равномерного распределения:
\[\bar{X} < Z_{tr} < Z_{R} < Z_{Q} < med X\]


\section{Литература}
Максимов Ю. Д. Математическая статистика //СПб.: СПбГПУ. – 2004.

Гурский Е. И. Теория вероятностей с элементами математической статистики. – Высш. школа, 1971.

\section{Приложения}

Репозиторий с кодом программы и кодом отчёта: \href{https://github.com/unjamini/math-stat-lab}{https://github.com/unjamini/math-stat-lab}



